%-------------------------------------------------------------------------------
%                      PREAMBLE - Document Declaration
%-------------------------------------------------------------------------------
\documentclass[a4paper, leqno, 12pt]{article}

\usepackage{setspace}
\usepackage[asymmetric]{geometry}
\usepackage{amsmath}
\usepackage{graphicx}
\usepackage{graphics}
\usepackage{calc}
\usepackage{xintexpr}
% \usepackage{l3regex}
\usepackage{expl3}
\usepackage{pgfplots}
\usepackage{amsmath}
\usepackage{multicol}
\pgfplotsset{compat=1.16}


\geometry{hmargin=2.3cm,
vmargin=2.5cm }

\newenvironment{top_enumerate}{
\begin{enumerate}
  \setlength{\itemsep}{2em}
  \setlength{\topsep}{-0pt}
  \setlength{\partopsep}{-0pt}
}{\end{enumerate}}

\newlength{\EqL}
\newlength{\RunL}
\setlength{\RunL}{0pt}
\newcommand{\EqContent}{foo}
\renewcommand{\theequation}{\alph{equation}}

\ExplSyntaxOn
\tl_new:N \l_mathexp_tl
\cs_new:Npn \nodollar #1 {
    \tl_set:Nn \l_mathexp_tl {#1}
    \regex_replace_all:nnN { \$ } {} \l_mathexp_tl
    \tl_use:N \l_mathexp_tl
}
\ExplSyntaxOff

%-------------------------------------------------------------------------------
%                               DOCUMENT
%-------------------------------------------------------------------------------
\begin{document}
\singlespacing
\begin{center}
\textbf{
ECON101 - Maths seminar material - Week 1\\
\bigskip
Algebra
}
\end{center}
\bigskip

\begin{top_enumerate}

\item Given the following pair of coordinate points $(x,y)$, find and sketch the linear equation $y = ax + b$. Where necessary, make sure you enter any rational number as a fraction, and not as a decimal number.
 
\setcounter{equation}{0}  % reset counter 
\begin{enumerate}
	\setlength{\topsep}{-0pt}
	\setlength{\partopsep}{-0pt}
	\setlength{\itemsep}{10pt}
			\item $A= ({-2},{-1}),\, B= ({9},{-4})$
	 \quad \textbf{[3]}
\end{enumerate}\item Given \(A=\left( {\begin{array}{cc}
   {9} & {8} \\
   {2} & {-5} \\
 \end{array} } \right) \), \(B=\left( {\begin{array}{cc}
     {-1} & {2} \\
     {1} & {1} \\
    \end{array} } \right) \), calculate the following matrix operations. Where necessary, make sure you keep any rational number as a fraction, and not as a decimal number.
 \\ 
\setcounter{equation}{0}  % reset counter 
\setlength{\RunL}{0pt}
	\renewcommand{\EqContent}{\nodollar{$A+B$
	\qquad\textbf{[1]}}}
	\settowidth{\EqL}{$\qquad\EqContent\qquad$}
	\setlength{\RunL}{\RunL+\EqL}
	\xintifboolexpr { \RunL > 0.667*\textwidth }
		{\setlength{\RunL}{0pt}
		\\}
		{}
	\begin{minipage}{\EqL}
	\begin{equation}
	\EqContent
	\end{equation}
	\end{minipage}
	\renewcommand{\EqContent}{\nodollar{${2}A-{3}B$
	\qquad\textbf{[1]}}}
	\settowidth{\EqL}{$\qquad\EqContent\qquad$}
	\setlength{\RunL}{\RunL+\EqL}
	\xintifboolexpr { \RunL > 0.667*\textwidth }
		{\setlength{\RunL}{0pt}
		\\}
		{}
	\begin{minipage}{\EqL}
	\begin{equation}
	\EqContent
	\end{equation}
	\end{minipage}
	\renewcommand{\EqContent}{\nodollar{${5}A+{2}B$
	\qquad\textbf{[1]}}}
	\settowidth{\EqL}{$\qquad\EqContent\qquad$}
	\setlength{\RunL}{\RunL+\EqL}
	\xintifboolexpr { \RunL > 0.667*\textwidth }
		{\setlength{\RunL}{0pt}
		\\}
		{}
	\begin{minipage}{\EqL}
	\begin{equation}
	\EqContent
	\end{equation}
	\end{minipage}
	\renewcommand{\EqContent}{\nodollar{$AB$
	\qquad\textbf{[3]}}}
	\settowidth{\EqL}{$\qquad\EqContent\qquad$}
	\setlength{\RunL}{\RunL+\EqL}
	\xintifboolexpr { \RunL > 0.667*\textwidth }
		{\setlength{\RunL}{0pt}
		\\}
		{}
	\begin{minipage}{\EqL}
	\begin{equation}
	\EqContent
	\end{equation}
	\end{minipage}
	\renewcommand{\EqContent}{\nodollar{$|A|$
	\qquad\textbf{[2]}}}
	\settowidth{\EqL}{$\qquad\EqContent\qquad$}
	\setlength{\RunL}{\RunL+\EqL}
	\xintifboolexpr { \RunL > 0.667*\textwidth }
		{\setlength{\RunL}{0pt}
		\\}
		{}
	\begin{minipage}{\EqL}
	\begin{equation}
	\EqContent
	\end{equation}
	\end{minipage}
	\renewcommand{\EqContent}{\nodollar{$|B|$
	\qquad\textbf{[2]}}}
	\settowidth{\EqL}{$\qquad\EqContent\qquad$}
	\setlength{\RunL}{\RunL+\EqL}
	\xintifboolexpr { \RunL > 0.667*\textwidth }
		{\setlength{\RunL}{0pt}
		\\}
		{}
	\begin{minipage}{\EqL}
	\begin{equation}
	\EqContent
	\end{equation}
	\end{minipage}
	\renewcommand{\EqContent}{\nodollar{$A^{-1}$
	\qquad\textbf{[2]}}}
	\settowidth{\EqL}{$\qquad\EqContent\qquad$}
	\setlength{\RunL}{\RunL+\EqL}
	\xintifboolexpr { \RunL > 0.667*\textwidth }
		{\setlength{\RunL}{0pt}
		\\}
		{}
	\begin{minipage}{\EqL}
	\begin{equation}
	\EqContent
	\end{equation}
	\end{minipage}
	\renewcommand{\EqContent}{\nodollar{$B^{-1}$
	\qquad\textbf{[2]}}}
	\settowidth{\EqL}{$\qquad\EqContent\qquad$}
	\setlength{\RunL}{\RunL+\EqL}
	\xintifboolexpr { \RunL > 0.667*\textwidth }
		{\setlength{\RunL}{0pt}
		\\}
		{}
	\begin{minipage}{\EqL}
	\begin{equation}
	\EqContent
	\end{equation}
	\end{minipage}
\item Find the values of $x$ which solve the following equations. Note that $x_1 < x_2$ and make sure to enter your answers in square brackets, e.g.$[-5,3]$.
 
\setcounter{equation}{0}  % reset counter 
\begin{multicols}{2}
\begin{enumerate}
	\setlength{\topsep}{-0pt}
	\setlength{\partopsep}{-0pt}
	\setlength{\itemsep}{10pt}
			\item $-{2}x^2 + {10}x + {9} = 0$
	 \quad \textbf{[3]}
		\item $x^2 + {20}x + {11} = 0$
	 \quad \textbf{[3]}
\end{enumerate}\end{multicols}\item Solve the following pair of equations. Make sure to enter your answers in square brackets, e.g. $[5,-3]$.
 
\setcounter{equation}{0}  % reset counter 
\begin{enumerate}
	\setlength{\topsep}{-0pt}
	\setlength{\partopsep}{-0pt}
	\setlength{\itemsep}{10pt}
			\item $\left\{\begin{aligned}
	{16}x - {3}y & = {50}\\
	{8}x + {2}y & = {4}\\
	\end{aligned}\right.$
	 \quad \textbf{[2]}
\end{enumerate}

\end{top_enumerate}
%-------------------------------------------------------------------------------
\end{document}
%-------------------------------------------------------------------------------
