%-------------------------------------------------------------------------------
% Template setup -- DO NOT MODIFY
\documentclass[preview]{standalone}
\usepackage{amsmath}
\usepackage{pgfplots}
\pgfplotsset{compat=1.16}
\newcommand \fieldname[1]{\underline{\texttt{#1}:}}
\begin{document}
%-------------------------------------------------------------------------------
% Overall exercise setup - main options and header
\fieldname{category}  % Classification information
Statistics

\fieldname{subCategory} % Classification information
Probability distributions

\fieldname{questionFormat}
Vlist

\fieldname{feedbackFormat}
Vlist

\fieldname{exerciseParameters}
% Random draws from list of 'well-behaved' values of the standard normal
L1:[0.5, 1, 1.25]
L2:[1.645, 1.96, 2.575]
idx1:1+rand(3)
idx2:1+rand(3)
idx3:1+rand(3)
idx4:1+rand(3)
idx5:1+rand(3)

r2:L2[idx1]
r1:-r2
r7:L2[idx2]
r4:L1[idx3]
r3:-r4
r5:L1[idx4]
r6:L2[idx5]

% Corresponding probabilities as drawn from the P(0 < Z < z_c) table
p1:norm_cdf(r2,0,1)-0.5
p2:norm_cdf(r7,0,1)-0.5
p3:norm_cdf(r4,0,1)-0.5
p5:norm_cdf(r5,0,1)-0.5
p6:norm_cdf(r6,0,1)-0.5

% Solutions to questions.
solA:round(2*p1*1000)/1000
solB:round(500-p2*1000)/1000
solC:round(2*p3*1000)/1000
solD:round(p6*1000-p5*1000)/1000

\fieldname{decimalPlaces}
3

\fieldname{questionBlurb}
Using the table of areas for the standard Normal distribution, calculate the areas under $N(0,1)$, rounded to 3 decimal places:

\fieldname{feedbackBlurb}
Solutions:

%-------------------------------------------------------------------------------
% This section contains the individual question information
\fieldname{questionType}
Numerical

\fieldname{questionText}
Between {r1} and {r2}

\fieldname{prompt}
$P({r1} \le Z \le {r2})$

\fieldname{solution}
solA

\fieldname{tolerance}
0.001

\fieldname{mark}
2

\fieldname{feedbackText}
\[
\begin{aligned}
P({r1} \le Z \le {r2}) & = P({r1} \le Z \le 0 ) + P( 0 \le Z \le {r2})\\
P({r1} \le Z \le {r2}) & = 2 \times P( 0 \le Z \le {r2})\\
P({r1} \le Z \le {r2}) & = 2 \times {p1} = {solA}\\
\end{aligned}
\]

%-------------------------------------------------------------------------------
% This section contains the individual question information
\fieldname{questionType}
Numerical

\fieldname{questionText}
For values greater than {r7}

\fieldname{prompt}
$P(Z \ge {r7} )$

\fieldname{solution}
solB

\fieldname{tolerance}
0.001

\fieldname{mark}
2

\fieldname{feedbackText}
\[
\begin{aligned}
P(Z \ge {r7}) = 0.5 - P( 0 \le Z \le {r5})\\
P(Z \ge {r7}) = 0.5 - {p2} = {solB} \\
\end{aligned}
\]

%-------------------------------------------------------------------------------
% This section contains the individual question information
\fieldname{questionType}
Numerical

\fieldname{questionText}
Between {r3} and {r4}

\fieldname{prompt}
$P({r3} \le Z \le {r4} )$

\fieldname{solution}
solC

\fieldname{tolerance}
0.001

\fieldname{mark}
2

\fieldname{feedbackText}
\[
\begin{aligned}
P({r3} \le Z \le {r4}) & = P({r3} \le Z \le 0 ) + P( 0 \le Z \le {r4})\\
P({r3} \le Z \le {r4}) & = 2 \times P( 0 \le Z \le {r4})\\
P({r3} \le Z \le {r4}) & = 2 \times {p3} = {solC}\\
\end{aligned}
\]

%-------------------------------------------------------------------------------
% This section contains the individual question information
\fieldname{questionType}
Numerical

\fieldname{questionText}
Between {r5} and {r6}.

\fieldname{prompt}
$P({r5} \le Z \le {r6})$

\fieldname{solution}
solD

\fieldname{tolerance}
0.001

\fieldname{mark}
2

\fieldname{feedbackText}
\[
\begin{aligned}
P({r5} \le Z \le {r6}) & = P( 0 \le Z \le {r6}) - P( 0 \le Z \le {r5})\\
P({r5} \le Z \le {r6}) & = {p6} - {p5} = {solD}\\
\end{aligned}
\]

%-------------------------------------------------------------------------------
\end{document}
