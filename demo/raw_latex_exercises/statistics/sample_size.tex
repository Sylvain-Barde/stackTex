%-------------------------------------------------------------------------------
% Template setup -- DO NOT MODIFY
\documentclass[preview]{standalone}
\usepackage{amsmath}
\usepackage{pgfplots}
\pgfplotsset{compat=1.16}
\newcommand \fieldname[1]{\underline{\texttt{#1}:}}
\begin{document}
%-------------------------------------------------------------------------------
% Overall exercise setup - main options and header
\fieldname{category}  % Classification information
Statistics

\fieldname{subCategory} % Classification information
Inference

\fieldname{questionFormat}
Vlist

\fieldname{feedbackFormat}
Vlist

\fieldname{exerciseParameters}
% Random index draw for fixed list of significance critical values
idx:1+rand(3)
L1:[90, 95, 99]
L2:[1.645, 1.96, 2.575]

c3:L1[idx]
c4:L2[idx]

% Random draws for standard deviation and required error
r3:5 + rand(11)
r4:1 + rand(30)/10

p1_a:c3/100
p1_b:p1_a/2

p3:c4*(r3/r4)
p4:p3^2

% Solution
solA: ceiling(p4)

\fieldname{decimalPlaces}
3

\fieldname{questionBlurb}

You are told that a random variable $X$ has a population standard deviation of {r3}.

% A researcher wishes to estimate the mean weekly wage of several thousands of workers employed in a plant within plus or minus £{r4} with a {c3}\% level of confidence. From past experience the researcher knows that the weekly wages are normally distributed with a standard deviation of £{r3}.

\fieldname{feedbackBlurb}
Solution:
%-------------------------------------------------------------------------------
% This section contains the individual question information
\fieldname{questionType}
Numerical

\fieldname{questionText}
How large a sample $n$ would you need in order to be able to estimate the population mean to within $\pm${r4} at the {c3}\% confidence level? Round your answer up to an integer.

\fieldname{prompt}
$n$

\fieldname{solution}
solA

\fieldname{tolerance}
2

\fieldname{mark}
2

\fieldname{feedbackText}
In order to determine how large a sample we need to achieve an error of $\pm${r4} at {c3}\% confidence, we can invert the formula for the confidence interval, where $z_{{c3}}$ is the critical value of the standard normal distribution. Suppose we pick the upper $U_{{c3}}$ bound of the confidence interval:

\[
\begin{aligned}
U_{{c3}} & = \bar x + z_{{c3}} \frac{\sigma}{\sqrt{n}}\\
U_{{c3}} - \bar x  & = z_{{c3}} \frac{\sigma}{\sqrt{n}}\\
\sqrt{n} & =z_{{c3}} \frac{\sigma}{U_{{c3}} - \bar x} \\
n & = \left(z_{{c3}} \frac{\sigma}{U_{{c3}} - \bar x}\right)^2 \\
\end{aligned}
\]

The {c3}\% critical value interval can be constructed using the standard normal distribution, but looking for the value of $z_{{c3}}$ such that:
\[
P(-z_{{c3}} < Z < z_{{c3}}) = {p1_a} \quad \iff \quad P(0 \le Z \le z_{{c3}}) = {p1_b}
\]
Using the table of values, we find $z_{{c3}} = {c4}$. Replacing this in the equation above, and including the information that the upper bound of the confidence interval should be {r4} above the mean, we can find the required sample size.

\[
n = \left({c4} \frac{{r3}}{{r4}}\right)^2 \approx \left({p3}\right)^2 \approx {p4}
\]
We need a sample of at least {solA} observations in order to generate a confidence interval of $\pm${r4} around the sample mean.

%-------------------------------------------------------------------------------
\end{document}
