%-------------------------------------------------------------------------------
% Template setup -- DO NOT MODIFY
\documentclass[preview]{standalone}
\usepackage{amsmath}
\usepackage{pgfplots}
\pgfplotsset{compat=1.16}
\newcommand \fieldname[1]{\underline{\texttt{#1}:}}
\begin{document}
%-------------------------------------------------------------------------------
% Overall exercise setup - main options and header
\fieldname{category}  % Classification information
Statistics

\fieldname{subCategory} % Classification information
Probability distributions

\fieldname{questionFormat}
Vlist

\fieldname{feedbackFormat}
Vlist

\fieldname{exerciseParameters}
% Part a, random sample size and observed successes
n0: 6 + rand(5)
n2: 1 + rand(3)
n1: n0 - n2

% Part b, random sample size and observed successes
n3: 5 + rand(5)
n4: 1 + rand(3)
n5: n3 - n4

% True probabilities of success
pSa: 1/2
pSb: 1/6

% Binomial probabilities
p0: binom_pdf(n1,n0,pSa)
p1: binom_pdf(n4,n3,pSb)

% Solutions, rounded to 4 decimal places
solA:round(10000*p0)/10000
solB:round(10000*p1)/10000

\fieldname{decimalPlaces}
4

\fieldname{questionBlurb}
What is the probability, rounded to 4 decimal places:

\fieldname{feedbackBlurb}
Solutions:

%-------------------------------------------------------------------------------
% This section contains the individual question information
\fieldname{questionType}
Numerical

\fieldname{questionText}
Of getting exactly {n1} heads and {n2} tails in {n0} tosses of a fair coin?

\fieldname{prompt}
$P(H = {n1})$

\fieldname{solution}
solA

\fieldname{tolerance}
0.001

\fieldname{mark}
3

\fieldname{feedbackText}

The probability of getting $H$ heads can be calculated using a Binomial distribution, with $N={n0}$ being the number of trials and $p=1/2$ being the probability of a success (getting a heads):
\[
P(H = {n1})  = C^{{n0}}_{{n1}}  \left(\frac{1}{2}\right)^{{n0}} \approx {solA}
\]
Note that we define $H$ as the number of heads, however one could chose instead to count the number of tails. The binomial formula should provide the same answers in both cases, as $C^{{n0}}_{{n1}} = C^{{n0}}_{{n2}}$ and $p=1-p=1/2$. Hopefully this is intuitive, the probability of the event should be the same whether you count heads or tails.

%-------------------------------------------------------------------------------
% This section contains the individual question information
\fieldname{questionType}
Numerical

\fieldname{questionText}
Of getting {n4} sixes in {n3} rolls of a fair dice?

\fieldname{prompt}
$P(X = {n4})$

\fieldname{solution}
solB

\fieldname{tolerance}
0.001

\fieldname{mark}
3

\fieldname{feedbackText}

The probability of getting $X$ sixes can be calculated using a Binomial distribution, with $N={n0}$ being the number of trials and $p=1/6$ being the probability of a success (getting a six):
\[
P(X = {n4})  = C^{{n3}}_{{n4}}  \left(\frac{1}{6}\right)^{{n4}} \left(\frac{5}{6}\right)^{{n5}} \approx {solB}
\]


%-------------------------------------------------------------------------------
\end{document}
