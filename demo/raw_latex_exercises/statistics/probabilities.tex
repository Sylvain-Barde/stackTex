%-------------------------------------------------------------------------------
% Template setup -- DO NOT MODIFY
\documentclass[preview]{standalone}
\usepackage{amsmath}
\usepackage{pgfplots}
\pgfplotsset{compat=1.16}
\newcommand \fieldname[1]{\underline{\texttt{#1}:}}
\begin{document}
%-------------------------------------------------------------------------------
% Overall exercise setup - main options and header
\fieldname{category}  % Classification information
Statistics

\fieldname{subCategory} % Classification information
Probability and descriptive stats

\fieldname{questionFormat}
Vlist

\fieldname{feedbackFormat}
Vlist

\fieldname{exerciseParameters}

% Random draws for underlying probabilities
pa:(25+rand(11))/100
pb:(65+rand(11))/100
pAB:(75+rand(11))/100

% Derived probabilities for solution/display
p1: pa + pb - pAB
p2: pa*pb
p3: pa + pb - pAB - pa*pb

% Conditional solution for Q(a)
solA: if round(p3*100) == 0 then true else false
txt1: if round(p3*100) == 0 then "=" else "\\ne"
txt2: if round(p3*100) == 0 then "" else "not "

\fieldname{decimalPlaces}
2

\fieldname{questionBlurb}
You are told $P(A) = {pa}$, $P(B) = {pb}$ and $P(A \textrm{ or } B) = {pAB}$. Are the following statements true or false? Provide an explanation in each case.

\fieldname{feedbackBlurb}
Solutions:

%-------------------------------------------------------------------------------
% This section contains the individual question information
\fieldname{questionType}
Algebraic

\fieldname{questionText}
$A$ and $B$ are independent.

\fieldname{prompt}
$A$ and $B$ independent?

\fieldname{solution}
solA

\fieldname{tolerance}


\fieldname{mark}
3

\fieldname{feedbackText}
By the addition rule we know that $P(A \textrm{ and } B) = P(A) + P(B) - P(A \textrm{ or } B)$. Using the values above, this give us:
\[
P(A \textrm{ and } B) = {pa} + {pb} - {pAB} = {p1}
\]
Furthermore, we can calculate:
\[
P(A)\times P(B) = {pa} \times {pb} = {p2}
\]
Because $P(A \textrm{ and } B) \, {txt1} \, P(A)\times P(B)$, one can conclude that $A$ and $B$ are {txt2}independent. \emph{{solA}}

%-------------------------------------------------------------------------------
% This section contains the individual question information
\fieldname{questionType}
Algebraic

\fieldname{questionText}
$A$ and $B$ are mutually exclusive.

\fieldname{prompt}
$A$ and $B$ mutually exclusive?

\fieldname{solution}
false

\fieldname{tolerance}


\fieldname{mark}
3

\fieldname{feedbackText}
$A$ and $B$ cannot be mutually exclusive, as $P(A \textrm{ and } B) = {p1} \ne 0$. There is a probability that both events will happen together. \emph{False}

%-------------------------------------------------------------------------------
% This section contains the individual question information
\fieldname{questionType}
Algebraic

\fieldname{questionText}
If two events that occur with non-zero probabilities are mutually exclusive, then they cannot be independent.

\fieldname{prompt}
Mutually exclusive $\implies$ not independent?

\fieldname{solution}
true

\fieldname{tolerance}


\fieldname{mark}
4

\fieldname{feedbackText}
If two events $A$ and $B$ are mutually exclusive, then $P(A \textrm{ and } B) = 0$. If these two events are also independent, then  we also have $P(A \textrm{ and } B) = P(A)\times P(B)$. The combination of the two implies:
\[
P(A \textrm{ and } B) = P(A)\times P(B) = 0
\]
This can only be the case if at least one of the events has a zero probability. As we are told that both events have non-zero probability, they cannot be mutually exclusive and independent. \emph{True}.

%-------------------------------------------------------------------------------
\end{document}
