%-------------------------------------------------------------------------------
% Template setup -- DO NOT MODIFY
\documentclass[preview]{standalone}
\usepackage{amsmath}
\usepackage{pgfplots}
\pgfplotsset{compat=1.16}
\newcommand \fieldname[1]{\underline{\texttt{#1}:}}
\begin{document}
%-------------------------------------------------------------------------------
% Overall exercise setup - main options and header
\fieldname{category}  % Classification information
Statistics

\fieldname{subCategory} % Classification information
Inference

\fieldname{questionFormat}
Vlist

\fieldname{feedbackFormat}
Vlist

\fieldname{exerciseParameters}
r1:40+rand(21)
r2:70+rand(21)
r3:25 + rand(10)

p1:(r3^2)/r1
p2:sqrt(p1)

cl_1:r2 - 1.645*round(p2*100)/100
cr_1:r2 + 1.645*round(p2*100)/100

cl_2:r2 - 1.96*round(p2*100)/100
cr_2:r2 + 1.96*round(p2*100)/100

cl_3:r2 - 2.575*round(p2*100)/100
cr_3:r2 + 2.575*round(p2*100)/100

solA: [cl_1, cr_1]
solB: [cl_2, cr_2]
solC: [cl_3, cr_3]

\fieldname{decimalPlaces}
2

\fieldname{questionBlurb}
A random sample of size $n={r1}$ is taken from a large population with known standard deviation $\sigma = {r3}$. If the sample mean is {r2}, calculate to 2 decimal places the following confidence intervals for the population mean $\mu$:

\fieldname{feedbackBlurb}
Solution:

%-------------------------------------------------------------------------------
% This section contains the individual question information
\fieldname{questionType}
Numerical

\fieldname{questionText}
The 90\% confidence interval.

\fieldname{prompt}
$C.I._{90}$

\fieldname{solution}
solA

\fieldname{tolerance}
0.1

\fieldname{mark}
3

\fieldname{feedbackText}
If the sample of size $n={r1}$ has a mean $\bar x = {r2}$ and the population has a standard deviation $\sigma={r3}$, then the sampling distribution of the mean follows a normal distribution:
\[
\bar X \sim N \left(\bar x,\frac{\sigma^2}{n}\right) = N \left({r2},\frac{{r3}^2}{{r1}}\right) = N \left({r2},{p1}\right)
\]
Note, we are given the population standard deviation $\sigma$, which means that we can use the Normal distribution to calculate the confidence interval regardless of sample size (i.e. we do not need the Central Limit Theorem).

The 90\% confidence interval can be constructed using the standard normal distribution, but looking for the value of $z$ such that:
\[
P(-z < Z < z) = 0.9 \quad \iff \quad P(0 \le Z \le z) = 0.45
\]
Using the table of values, we find $z = \pm 1.645$. Using the mean and standard deviation of the distribution of $\bar X$, we can then recover 90\% confidence interval.
\[
C.I._{90} = \bar x \pm 1.645 \frac{\sigma}{\sqrt{n}} = {r2} \pm 1.645 \cdot {p2} = ({cl_1}, {cr_1})
\]
With 90\% confidence, the population mean $\mu$ is between {cl_1} and {cr_1}.

%-------------------------------------------------------------------------------
% This section contains the individual question information
\fieldname{questionType}
Numerical

\fieldname{questionText}
The 95\% confidence interval.

\fieldname{prompt}
$C.I._{95}$

\fieldname{solution}
solB

\fieldname{tolerance}
0.1

\fieldname{mark}
3

\fieldname{feedbackText}
The 95\% confidence interval can be constructed by looking for the value of $z$ such that:
\[
P(-z < Z < z) = 0.95 \quad \iff \quad P(0 \le Z \le z) = 0.475
\]
Using the table of values, we find $z = \pm 1.96$. Using the mean and standard deviation of the distribution of $\bar X$, we can then recover 95\% confidence interval.
\[
C.I._{95} = \bar x \pm 1.96 \frac{\sigma}{\sqrt{n}} = {r2} \pm 1.96 \cdot {p2} = ({cl_2}, {cr_2})
\]
With 95\% confidence, the population mean $\mu$ is between {cl_2} and {cr_2}.

%-------------------------------------------------------------------------------
% This section contains the individual question information
\fieldname{questionType}
Numerical

\fieldname{questionText}
The 99\% confidence interval.

\fieldname{prompt}
$C.I._{99}$

\fieldname{solution}
solC

\fieldname{tolerance}
0.1

\fieldname{mark}
3

\fieldname{feedbackText}
The 99\% confidence interval can be constructed by looking for the value of $z$ such that:
\[
P(-z < Z < z) = 0.99 \quad \iff \quad P(0 \le Z \le z) = 0.495
\]
Using the table of values, we find $z = \pm 2.575$. Using the mean and standard deviation of the distribution of $\bar X$, we can then recover 99\% confidence interval.
\[
C.I._{99} = \bar x \pm 2.575 \frac{\sigma}{\sqrt{n}} = {r2} \pm 2.575 \cdot {p2} = ({cl_3}, {cr_3})
\]
With 99\% confidence, the population mean $\mu$ is between {cl_3} and {cr_3}.

%-------------------------------------------------------------------------------
\end{document}
