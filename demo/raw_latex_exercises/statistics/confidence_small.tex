%-------------------------------------------------------------------------------
% Template setup -- DO NOT MODIFY
\documentclass[preview]{standalone}
\usepackage{amsmath}
\usepackage{pgfplots}
\pgfplotsset{compat=1.16}
\newcommand \fieldname[1]{\underline{\texttt{#1}:}}
\begin{document}
%-------------------------------------------------------------------------------
% Overall exercise setup - main options and header
\fieldname{category}  % Classification information
Statistics

\fieldname{subCategory} % Classification information
Inference

\fieldname{questionFormat}
Vlist

\fieldname{feedbackFormat}
Vlist

\fieldname{exerciseParameters}
% Draw 3 random sample sizes and corresponding (n-1) t-stats at 95%
idx:1+rand(3)
L1: [9, 16, 25]
L2: [2.306, 2.131, 2.064]

p1: L1[idx]
p2: 70+rand(21)
p3: 25 + rand(10)
p4: p3/sqrt(p1)
p5: L2[idx]

[cl1,cr1]: [p2 - round(1.96*p4*100)/100, p2 + round(1.96*p4*100)/100]
[cl2,cr2]: [p2 - round(p5*p4*100)/100, p2 + round(p5*p4*100)/100]
solA: [cl2,cr2]
solB: [cl1,cr1]

\fieldname{decimalPlaces}
3

\fieldname{questionBlurb}
A random sample of size $n={p1}$ is taken from a large population. The sample mean is $\bar x = {p2}$, and the sample standard deviation is $s = {p3}$.

\fieldname{feedbackBlurb}
Solution:

%-------------------------------------------------------------------------------
% This section contains the individual question information
\fieldname{questionType}
Numerical

\fieldname{questionText}
Using this information, construct a 95\% confidence interval for the population mean $\mu$. Enter the interval using square brackets, separating the two values with commas $\left[\dots ,\dots \right]$:

\fieldname{prompt}
$C.I._{95}$

\fieldname{solution}
solA

\fieldname{tolerance}
0.1

\fieldname{mark}
2

\fieldname{feedbackText}
Given that the sample has a size $n={p1}$, if we do not know the population standard deviation $\sigma$, then we need to (a) replace $\sigma$ by the sample standard deviation $s$ and (b) rely on the $t$-distribution instead of the standard normal distribution, to allow for additional uncertainty generated by using $s$, which is an estimate of $\sigma$:
\[
C.I._{95} = \bar x \pm t_{0.05}^{n-1} \frac{s}{\sqrt{n}}
\]

The critical value $t_{0.05}^{n-1}$ can be found in the table of values for Student's t-distribution, for a two-tailed probability of 0.05 (corresponding to 95\% confidence) and $n-1 = {{p1}-1}$ degrees of freedom. In this case, $t_{0.05}^{{p1}-1} = {p5}$. This gives us the following confidence interval:
\[
C.I._{95} =  {p2} \pm {p5}\frac{{p3}}{\sqrt{{p1}}}  =  \left[{cl2} \, , \, {cr2}\right]
\]

%-------------------------------------------------------------------------------
% This section contains the individual question information
\fieldname{questionType}
Numerical

\fieldname{questionText}
Suppose you are told that {p3} is also the value of the population standard deviation $\sigma$. Construct a new 95\% confidence interval and compare this with your previous calculation.

\fieldname{prompt}
$C.I._{95}$

\fieldname{solution}
solB

\fieldname{tolerance}
0.1

\fieldname{mark}
3

\fieldname{feedbackText}
If the sample of size $n={p1}$ has a mean $\bar x = {p2}$ and the population has standard deviation $\sigma={p3}$, then the sampling distribution of the mean follows a normal distribution:
\[
\bar X \sim N \left(\bar x,\frac{\sigma^2}{n}\right) = N \left({p2},\frac{{p3}^2}{{p1}}\right) \approx N \left({p2},{p4}^2\right)
\]
Note, because we now know the value of the population standard deviation $\sigma$, we are able to calculate confidence intervals using the normal distribution, and we do not need to use Student's t-distribution. This is only required when (a) we use the sample standard deviation $s$ as an estimate of $\sigma$ and (b) the sample size $n<30$, so that there is a lot of uncertainty in the accuracy of $s$.

The 95\% confidence interval can be constructed using the standard normal distribution, but looking for the value of $z$ such that:
\[
P(-z < Z < z) = 0.95 \quad \iff \quad P(0 \le Z \le z) = 0.475
\]
Using the table of values, we find $z = \pm 1.96$. Using the mean and standard deviation of the distribution of $\bar X$, we can then calculate 95\% confidence interval.
\[
C.I._{95} = \bar x \pm 1.96 \frac{\sigma}{\sqrt{n}} = {p2} \pm 1.96 \frac{{p3}}{\sqrt{{p1}}} = \left[{cl1} \, , \, {cr1}\right]
\]
This is much tighter confidence interval than the previous answer. Even if the actual value of the standard deviation has not changed (still $p3$), the fact that we know for certain the value of the population standard deviation $\sigma$ rather than have to rely on an estimate $s$ means we can be much more precise for a given confidence level.

%-------------------------------------------------------------------------------
\end{document}
