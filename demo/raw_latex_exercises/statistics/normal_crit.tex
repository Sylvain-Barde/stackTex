%-------------------------------------------------------------------------------
% Template setup -- DO NOT MODIFY
\documentclass[preview]{standalone}
\usepackage{amsmath}
\usepackage{pgfplots}
\pgfplotsset{compat=1.16}
\newcommand \fieldname[1]{\underline{\texttt{#1}:}}
\begin{document}
%-------------------------------------------------------------------------------
% Overall exercise setup - main options and header
\fieldname{category}  % Classification information
Statistics

\fieldname{subCategory} % Classification information
Probability distributions

\fieldname{questionFormat}
Vlist

\fieldname{feedbackFormat}
Vlist

\fieldname{exerciseParameters}
% Random draw for 3 probabilities
r1:0.01+rand(9)/100
r2:0.2+rand(21)/100
r3:0.4+rand(21)/100

% Transforms to fit the P(0 < Z < z_c) CDF table for the standard normal
p1:0.5-r1
p2:0.5-r2
p3:r3/2

% Quantiles for the normal CDF
c1:norm_qnt(p1+0.5,0,1)
c2:norm_qnt(p2+0.5,0,1)
c3:norm_qnt(p3+0.5,0,1)

% Rounded solutions
solA:round(c1*100)/100
solB:-round(c2*100)/100
solC:round(c3*100)/100

\fieldname{decimalPlaces}
3

\fieldname{questionBlurb}
If $Z \sim N(0,1)$, find the critical value $z_0$, rounded to 2 decimal places, such that:

\fieldname{feedbackBlurb}
Solutions:

Note: It may help you to draw pictures to visualise the area you are being asked to calculate.

%-------------------------------------------------------------------------------
% This section contains the individual question information
\fieldname{questionType}
Numerical

\fieldname{questionText}
$P(Z \ge z_0) = {r1}$

\fieldname{prompt}
$z_0$

\fieldname{solution}
solA

\fieldname{tolerance}
0.01

\fieldname{mark}
2

\fieldname{feedbackText}
If $P(Z \ge z_0) = {r1}$, then $P(0 \le Z \le z_0)  = 0.5 - P(z \ge z_0) = {p1}$. Looking this value up in the table for the standard normal tells us that $z_0 = {c1}$.

%-------------------------------------------------------------------------------
% This section contains the individual question information
\fieldname{questionType}
Numerical

\fieldname{questionText}
$P(Z \le z_0) = {r2}$

\fieldname{prompt}
$z_0$

\fieldname{solution}
solB

\fieldname{tolerance}
0.01

\fieldname{mark}
2

\fieldname{feedbackText}
If $P(Z \le z_0) = {r2}$, then $P(z_0 \le z \le 0) = 0.5 - P(Z \le z_0) = {p2}$. In the tables of the standard distribution, this corresponds to a value of ${c2}$. However, we are looking for a value of $z$ that is smaller than $z_0$, therefore to the left of it. Given that the probability is less than 0.5, the critical value $z_0$ must be negative. We therefore have $z_0 = -{c2}$. A diagram can be particularly useful for cases like this.

%-------------------------------------------------------------------------------
% This section contains the individual question information
\fieldname{questionType}
Numerical

\fieldname{questionText}
$P(-z_0 \le Z \le z_0) = {r3}$

\fieldname{prompt}
$z_0$

\fieldname{solution}
solC

\fieldname{tolerance}
0.01

\fieldname{mark}
2

\fieldname{feedbackText}
If $P(-z_0 \le Z \le z_0) = {r3}$ then $P(0 \le Z \le z_0) = \frac{{r3}}{2} = {p3}$. Looking up this value in the table tells us that $z_0 = {c3}$.

%-------------------------------------------------------------------------------
\end{document}
