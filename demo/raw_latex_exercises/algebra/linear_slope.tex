%-------------------------------------------------------------------------------
% Template setup -- DO NOT MODIFY
\documentclass[preview]{standalone}
\usepackage{amsmath}
\usepackage{pgfplots}
\pgfplotsset{compat=1.16}
\newcommand \fieldname[1]{\underline{\texttt{#1}:}}
\begin{document}
%-------------------------------------------------------------------------------
% Overall exercise setup - main options and header
\fieldname{category}  % Classification information
Mathematics

\fieldname{subCategory} % Classification information
Algebra

\fieldname{questionFormat}
Vlist

\fieldname{feedbackFormat}
Vlist

\fieldname{exerciseParameters}
% Question A
[qAr1,qAr2,qAt3,qAr4]:[-rand(10),rand(10)*rand([-1,1]),rand(10),rand(10)*rand([-1,1])]
% Check if 'X' draws are the same, if so modify one of them
qAr3:if qAt3 == qAr1 then qAt3+rand([-2,-1,1,2]) else qAt3
% Calculate derived slope/interecpt parameters
[qAp1,qAp2]:[qAr4-qAr2,qAr3-qAr1]
qAp3:qAp2*qAr2-qAp1*qAr1
% Rounded slope parameters, with conditional text for sign of slope
[qAp5,qAp6]:[qAp3/qAp2,qAp1/qAp2]
qAsig:if qAp6 > 0 then "+" else "-"

\fieldname{questionBlurb}
Given the following pair of coordinate points $(x,y)$, find and sketch the linear equation $y = ax + b$. Where necessary, make sure you enter any rational number as a fraction, and not as a decimal number.

\fieldname{feedbackBlurb}
Solution:

%-------------------------------------------------------------------------------
% This section contains the individual question information
\fieldname{questionType}
Algebraic

\fieldname{questionText}
$A= ({qAr1},{qAr2}),\, B= ({qAr3},{qAr4})$

\fieldname{prompt}
$y=$

\fieldname{solution}
$\frac{{qAp3}}{{qAp2}}+\frac{{qAp1}}{{qAp2}}x$

\fieldname{tolerance}


\fieldname{mark}
3

\fieldname{feedbackText}
First, the slope is linear, which means that $a = \frac{y_B-y_A}{x_B-x_A}$. Once $a$ is known, its value can be replaced in the linear expression $y = ax + b$, and the $x$/$y$ values of either point can be used to determine the value of $b$.

For $A= ({qAr1},{qAr2}),\, B= ({qAr3},{qAr4})$:
\[
\left\{\begin{aligned}
a & = \frac{y_B-y_A}{x_B-x_A} = \frac{{qAr4}-{qAr2}}{{qAr3}-{qAr1}} = \frac{{qAp1}}{{qAp2}}\\
b & = y_A -ax_A = {qAr2} - \frac{{qAp1}}{{qAp2}}{qAr1} = \frac{{qAp3}}{{qAp2}} \\
 \end{aligned}\right.
\]

Therefore $y = \frac{{qAp3}}{{qAp2}} + \frac{{qAp1}}{{qAp2}}x \approx {qAp5} {qAsig} {qAp6}x$

\begin{tikzpicture}
\begin{axis}[
xlabel=$x$,
ylabel=$y$,
axis lines = center,
legend style={at={(0.5,-0.1)},anchor=north,draw=none,legend columns=-1}]
\addplot[domain=-10:10,color=blue,]{ {qAp3}/{qAp2} +{qAp1}/{qAp2}*x};
\addlegendentry{$y=ax+b$}
\addplot[mark=*]coordinates {({qAr1},{qAr2})};
\addlegendentry{$A$}
\addplot[mark=o]coordinates {({qAr3},{qAr4})};
\addlegendentry{$B$}
\end{axis}
\end{tikzpicture}

%-------------------------------------------------------------------------------
\end{document}
